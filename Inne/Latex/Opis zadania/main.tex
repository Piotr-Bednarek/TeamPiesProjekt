\documentclass[12pt, a4paper]{article}

% Pakiety kodowania i językowe
\usepackage[utf8]{inputenc}
\usepackage[T1]{fontenc}
\usepackage[polish]{babel}

% Pakiety do geometrii i 
\usepackage{geometry}
\usepackage{graphicx}
\usepackage{float}
\usepackage{hyperref}
\usepackage{listings}
\usepackage{xcolor}

% Konfiguracja geometrii strony
\geometry{
    a4paper,
    left=25mm,
    top=25mm,
    bottom=25mm,
    right=25mm
}

% Konfiguracja linków
\hypersetup{
    colorlinks=true,
    linkcolor=black,
    filecolor=magenta,      
    urlcolor=blue,
}

% Konfiguracja wyświetlania kodu
\lstset{
    basicstyle=\ttfamily\footnotesize,
    breaklines=true,
    captionpos=b,
    frame=single,
    numbers=left,
    numberstyle=\tiny,
    keywordstyle=\color{blue},
    commentstyle=\color{green!60!black},
    stringstyle=\color{purple}
}

% Dane dokumentu
\title{Układ sterowania położeniem piłki pingpongowej na belce}
\author{Piotr Bednarek \\ Jan Andrzejewski \\ Mateusz Banaszak}
\date{\today}

\begin{document}

\maketitle

\section{Wstęp}
Przedmiotem projektu jest układ automatycznej regulacji pozycji piłki pingpongowej na
pochylonej belce. System wykorzystuje serwo TowerPro MG946R do sterowania kątem nachylenia belki
oraz laserowy czujnik odległości Time-of-Flight VL53L0X do pomiaru aktualnego położenia
piłki. Mikrokontroler STM32F767ZI analizuje dane z czujnika i w czasie rzeczywistym
koryguje nachylenie belki, aby utrzymać piłkę w zadanej pozycji lub śledzić określoną
trajektorię. Układ stanowi klasyczny przykład zastosowania regulacji PID w systemie
niestabilnym, gdzie niewielkie zakłócenia mogą prowadzić do utraty kontroli nad obiektem.

\section{Opis sprzętu}
Opis wykorzystanego mikrokontrolera (STM32) oraz peryferiów (czujniki, moduły).

\begin{itemize}
    \item STM Nucleo F767ZI
    \item Servo TowerPro MG946R
    \item Grove - VL53LOX Timeof Flight I2C
\end{itemize}

\begin{figure}[H]
    \centering
    \includegraphics[width=0.5\linewidth]{img/plytka.png}
\end{figure}

\begin{figure}[H]
    \centering
    \includegraphics[width=0.5\linewidth]{img/sensor.png}
\end{figure}

\begin{figure}[H]
    \centering
    \includegraphics[width=0.5\linewidth]{img/servo.png}
\end{figure}

\section{Wymagania dodatkowe}
\begin{enumerate}
    \item Podział kodu źródłowego na moduły i dokumentacja kodu zgodna ze standardem dokumentacji.
    \item Wykorzystanie systemu kontroli wersji (Git) z publicznym repozytorium (GitHub).
    \item Dedykowana aplikacja desktopowa jako graficzny interfejs użytkownika
          zaprogramowana w Pythonie za pomocą nowoczesnych bibliotek.
          % \item Zapewnienie uchybu ustalonego na poziomie 1\% zakresu regulacji w stanie ustalonym za pomocą
          %   odpowiednio dostrojonego regulatora PID.
    \item Implementacja śledzenia sygnałów sterujących i pomiarowych za pomocą wykresów w aplikacji desktopowej (UI).
    \item Dodatkowe urządzenie wyjścia użytkownika za pomocą rzędu diod LED RGB do
          wizualizacji uchybu układu.
    \item Dodatkowe przyciski do manualnej zmiany kątu nachylenia belki lub wentylator wprowadzający
          zakłócenia do układu.
    \item Zapewnienie logow z procesu stabilizacji do odtworzenia przebiegow uchyby i wyjscia po przeprowadzeniu eksperymentu (CSV/TXT)
    \item Dodatkowe urządzenie wejścia użytkownika w postaci potencjometru do zmiany zadanej pozycji piłki na belce.
    \item Wykorzystanie sumy kontrolnej do wykrywania błędów transmisji w komunikacji szeregowej.
    \item Wykorzystanie systemu czasu rzeczywistego FreeRTOS.
\end{enumerate}

\section{Model/druk 3D}

\begin{figure}[H]
    \centering
    \includegraphics[width=0.65\linewidth]{img/rampa1.png}
\end{figure}

\begin{figure}[H]
    \centering
    \includegraphics[width=0.65\linewidth]{img/rampa2.png}
\end{figure}

\end{document}
